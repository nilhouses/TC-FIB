\documentclass{article}
\usepackage[utf8]{inputenc}
\usepackage[T1]{fontenc}
\usepackage[catalan]{babel}
\usepackage{amsmath, amssymb}
\usepackage{parskip}
\usepackage[left=2.5cm, right=2.5cm, top=2cm, bottom=2cm]{geometry}
\usepackage{tcolorbox} %rectangle

\begin{document}

\section*{\small 1.3 Argumenteu si són certes (amb una justificació) o falses (amb un contraexemple) les següents afirmacions en general}

\textbf{t) $L = L^2 \implies (L = L^*) \vee (L = \emptyset)$}

Certa. Demostrem-ho per inducció:

Volem demostrar que $L^n = L$ per a tot $n \geq 1$.

\begin{itemize}
    \item \textbf{Cas base ($n = 1$):} $L^1 = L$
    \item \textbf{Pas inductiu:} Suposem com hipòtesi d'inducció (HI) que $L^k = L$ per a algun $k \geq 1$. Aleshores:
    \[
    L^{k+1} = L^k \cdot L \overset{\text{HI}}{=} L \cdot L = L^2 = L
    \]
    Per tant, per inducció, $L^n = L$ per a tot $n \geq 1$.
\end{itemize}

Llavors:
\[
L^* = \bigcup_{n=0}^{\infty} L^n = L^0 \cup \bigcup_{n=1}^{\infty} L = \{\lambda\} \cup L
\]

Distingim dos casos principals:

\begin{itemize}
    \item \textbf{Cas 1:  $\lambda \in L$}:  
    Aleshores $\{\lambda\} \subseteq L$, i com hem demostrat que $L^n = L$ per a tot $n \geq 1$, tenim que
    \[
    L^* = L^0 \cup L^1 \cup L^2 \cup \dots = \{\lambda\} \cup L \cup L \cup \dots = L
    \]
    Per tant, $ L = L^* $.

    \item \textbf{Cas 2: $\lambda \notin L$}:
\[
\lambda \notin  L \wedge L^* = \{\lambda\} \cup L   \implies \{\lambda\} \subseteq L^* \implies L \neq L^* 
\]

Per tal que la conclusió $(L = L^*) \vee (L = \emptyset)$ sigui certa, i sabent que $L \neq L^* $, ha de ser que $L = \emptyset$. Comprovem, doncs,  si $L = \emptyset$ és consistent amb la nostra suposició inicial $L = L^2$ i $\lambda \notin L$.

\[
 L = \emptyset \implies \lambda \notin L \wedge L^2 = \emptyset \cdot \emptyset = \emptyset  \implies L = L^2
 \]

En resum,  $\lambda \notin L \implies L = \emptyset$ 
\end {itemize}

Així doncs, hem demostrat que:

\[
L = L^2 \Rightarrow (L = L^*) \vee (L = \emptyset)
\]
\begin{tcolorbox}
\textbf{NOTA:} Aquest exercici també es pot demostrar elegantment usant les propietats dels apartats \textbf{s)} i \textbf{q)}:
\begin{itemize}
    \item \textbf{s)} $ \lambda \in L \wedge L^2 \subseteq L \iff L = L^* $
    \item \textbf{q)} $ L \subseteq L^2 \iff \lambda \in L \vee L = \emptyset $
\end{itemize}

Com que $L = L^2$, aleshores $L \subseteq L^2$, i per la propietat \textbf{q)} se'n dedueix que:
\[
\lambda \in L \vee L = \emptyset
\]

En el primer cas, si \( \lambda \in L \), com que \( L = L^2 \), aleshores \( L^2 \subseteq L \), i per tant:
\[
\lambda \in L \wedge L^2 \subseteq L \overset{\text{s)}}{\Rightarrow} L = L^*
\]

En el segon cas, si \( L = \emptyset \), també es compleix la propietat.

Així, es conclou que:
\[
L = L^2 \Rightarrow (L = L^*) \vee (L = \emptyset)
\]
\end{tcolorbox}



\end{document}
