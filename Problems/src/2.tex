\documentclass{article}
\usepackage[utf8]{inputenc}
\usepackage[T1]{fontenc}
\usepackage[catalan]{babel}
\usepackage{amsmath, amssymb}
\usepackage{parskip}
\usepackage[left=2.5cm, right=2.5cm, top=2cm, bottom=2cm]{geometry}
\usepackage{tcolorbox} %rectangle
\usepackage{listings} % Per a la visualització de codi

% Configuració del paquet listings
% Una configuració per un pseudo-codi més genèric
% Permet que les ordres matemàtiques dins de $...$ es renderitzin.
\lstdefinestyle{pseudocode}{
    basicstyle=\ttfamily\small,
    numbers=left,
    numberstyle=\tiny\color{gray},
    stepnumber=1,
    numbersep=5pt,
    showstringspaces=false,
    tabsize=4,
    breaklines=true,
    % captionpos=b, % Aquesta línia ja no és necessària si no hi ha caption
    frame=single,
    breakatwhitespace=false,
    mathescape=true, % AQUEST ÉS CLAU! Permet expressions matemàtiques amb $...$
    morekeywords={for, in, if, not, return, new, while} % Paraules clau
}


\begin{document}

\section*{\small 2.20 Disenya un algorisme de cost raonable per trobar els
estats no accesibles d'un DFA d'entrada.}

Sigui $D = (Q, \Sigma, \delta, q_0, F )$ el DFA d'entrada, on:
\begin{itemize}
    \item $Q$, conjunt d'estats
    \item $\Sigma$, alfabet
    \item $\delta : Q \times \Sigma \to Q$, funció de transició
    \item $q_0$, estat inicial
    \item $F$, conjunt d'estats finals (o acceptadors)
\end{itemize}

\subsection*{\small Algorisme}
\begin{lstlisting}[style=pseudocode] % Hem eliminat 'caption=Algorisme per trobar estats no accessibles'
A = {$q_0$}
c = new Queue<>();
c.push($q_0$);
while (not c.empty()) {
    q = c.pop();
    for (s in $\Sigma$) {
        $q'$ = $\delta(q, s)$;
        if ($q' \notin A$) {
            A = A $\cup$ {$q'$};
            c.push($q'$);
        }
    }
}
return $Q \setminus A$
\end{lstlisting}

L'algorisme utilitza en un $BFS$ per trobar els estats accessibles i retornar els no accessibles.

\subsection*{\small Cost}

El cost serà el d'executar un $BFS$ al graf de l'autòmat, que té $|Q|$ nodes (o estats) i $|Q| \cdot |\Sigma|$ arestes (o transicions):
\[
O(D) = O(|Q| + |Q| \cdot |\Sigma|) \in O(|Q| \cdot |\Sigma|)
\]
\:
\begin{tcolorbox}
L'algorisme es podria millorar usant heurístiques o poda.
\end{tcolorbox}

\end{document}